%\documentclass{article}
\documentclass[orivec,dvips,10pt]{llncs}

\usepackage{epsfig}
\usepackage{path}
\usepackage{url}
\usepackage{amsmath,amssymb} 

\newenvironment{template}{\sffamily}

\setlength{\parindent}{0pt}
\setlength{\parskip}{1ex}

\usepackage{graphics,epsfig}
\usepackage{stmaryrd}

\input{macros.ltx}
\input{library.ltx}

\NatPackage

\newcommand{\Ivor}{\textsc{Ivor}}
\newcommand{\Funl}{\textsc{Funl}}
\newcommand{\Agda}{\textsc{Agda}}

\newcommand{\mysubsubsection}[1]{
\noindent
\textbf{#1}
}
\newcommand{\hdecl}[1]{\texttt{#1}}

\begin{document}

\title{\Ivor{}, a Proof Engine --- DRAFT}
\author{Edwin Brady}

\institute{School of Computer Science, \\
 University of St Andrews, St Andrews, Scotland. \\ \texttt{Email: eb@dcs.st-and.ac.uk}.\\
\texttt{Tel: +44-1334-463253}, \texttt{Fax: +44-1334-463278} \vspace{0.1in}
}
 
\maketitle

\begin{abstract}
Dependent type theory has several practical applications in the fields
of theorem proving, program verification and programming language
design. \Ivor{} is a Haskell library designed to allow easy extending
and embedding of a type theory based theorem prover in a Haskell
application. In this paper, I give an overview of the library and show
how it can be used to implement formal systems such as propositional
logic.  Furthermore, I sketch an implementation of a simple functional
programming language using the library; by using type theory as a core
representation, we can construct and evaluate terms and prove
correctness properties of those terms within the \remph{same}
framework, ensuring consistency of the implementation and the theorem prover.

\end{abstract}

\section{Introduction}

%\Ivor{} is a tactic-based theorem proving engine with a Haskell
%API. Unlike other systems such as \Coq{}~\cite{coq-manual} and
%Agda~\cite{agda}, the tactic engine is primarily intended to be used
%by programs, rather than a human operator. 

Type theory based theorem provers such as \Coq{}~\cite{coq-manual} and
Agda~\cite{agda} have been used as tools for verification of programs
(e.g.~\cite{leroy-compiler,why-tool,mckinna-expr}), extraction of
correct programs from proofs (e.g.~\cite{extraction-coq,large-extraction})
and formal proofs of mathematical properties
(e.g.~\cite{fta,four-colour}).  However, these tools are designed with a
human operator in mind; the interface is textual which makes it
difficult for an external program to interact with them. \Coq{}
provides an extraction mechanism~\cite{extraction-coq} which generates
ML or Haskell code from a proof term, but this does not allow the easy
\remph{construction} of proof terms by an external tool. It is also
extensible to some extent, for example using a domain specific
language for creating user tactics, but the result is difficult to
embed in an external program.

%More
%recently, dependent types have been incorporated into programming
%languages such as Cayenne~\cite{cayenne-icfp}, DML~\cite{xi-thesis} and
%\Epigram{}~\cite{view-left,epigram-afp}.

In contrast, the \Ivor{} library is designed to provide an
implementation of dependent type theory and tactics for proof and
program development to a Haskell application programmer, via a stable,
well-documented and lightweight (as far as possible) API. The goal is
to allow: i) easy embedding of theorem proving tools in a Haskell
application; and ii) easy extension of the theorem prover with
\remph{domain specific} tactics.

%% have been used for several
%% large practical applications, including correctness proofs for a
%% compiler~\cite{leroy-compiler} and a computer assisted proof of the
%% four colour theorem~\cite{four-colour}. 

\subsection{Motivating Examples}

There are several situations where a dependently typed proof framework
accessible as a library from Haskell programs could be useful.  For
each of these, by using an implementation of a well understood type
theory, we can be confident that the underlying framework is
sound. There is no need to prove that a language and proof system are
consistent with each other or that the proof framework is
sound.

\begin{description}
\item[Formal Systems] 
A formal system can be modelled in dependent type theory, and
properties of the system and derivations within the system can be
proved. A simple example is first order logic --- it is
straightforward to represent the connectives $\land$, $\lor$ and $\to$
in dependent type theory and use a theorem prover to prove logical
formulae.
Having an implementation of type theory and an interactive theorem
prover accessible as an API makes it easy to write tools for working
in a formal system, whether for educational or practical purposes.
In section \ref{example1}, I will give details of an implementation of
first order logic.

\item[Programming Languages] 
Dependent type theory can be used as the internal representation for a
functional programming language. The core language of the Glasgow
Haskell Compiler is \SystemF{}~\cite{core} --- dependent type theory
generalises this by allowing types to be parametrised over values.
One advantage of a pure functional language is that correctness
properties of programs in the language can be proven by equational
reasoning.  Some tools exist to assist with this,
e.g. Sparkle~\cite{sparkle} for Clean, or Cover~\cite{cover} for
translating Haskell into Agda~\cite{agda}. However a problem with such
tools is that they separate the language implementation from the
theorem prover; by using a common internal representation for both, we can
be confident that proofs use a faithful representation of the program
and that the theorem prover supports the entire language. In section
\ref{example2}, we will see how \Ivor{} can be used to implement a
language with a built-in theorem prover.

\item[Verified DSL Implementation]
We have previously used \Ivor{} to demonstrate the implementation of
a verified domain specific language~\cite{dtpmsp-gpce}. The idea is to
represent the abstract syntax tree of a DSL program as a dependent
type, and write an interpreter which guarantees that invariant
properties of the program are maintained. Using staging
annotations~\cite{multi-taha}, such an interpreter can be specialised
to a translator. We are continuing to explore these techniques in the
context of resource aware programming~\cite{dt-framework}.

\end{description}

%% Whatever the situation, domain specific tactics are often
%% required. For example, 
%% an implementation of a programming language with subtyping may require
%% a tactic for inserting coercions, or
%% a computer arithmetic system may require an
%% implementation of Pugh's Omega decision procedure~\cite{pugh-omega}.
%% \Ivor{}'s API is designed to make implementation of such tactics as
%% easy as possible.

%\subsection{Why Do We Need Another Theorem Prover?}

%Relationship to e.g. \Coq{}.


\section{The Type Theory, $\source$}

\renewcommand{\Vnil}{\DC{vnil}}
\renewcommand{\Vcons}{\DC{vcons}}

\subsection{The Core Calculus}

\label{corett}

The core type theory of \Ivor{} is a strongly normalising dependently
typed $\lambda$-calculus with inductive families~\cite{dybjer94},
similar to Luo's UTT~\cite{luo94}, the Calculus of Inductive
Constructions in \Coq{}~\cite{coq-manual}, or \Epigram{}'s
ETT~\cite{epireloaded}.  This language, which I call
$\source$~\cite{brady-thesis}, is an enriched lambda calculus, with
the usual reduction rules, and properties of subject reduction, Church
Rosser, and uniqueness of types up to conversion. The strong
normalisation property (i.e. that evaluation always terminates) is
guaranteed by allowing only primitive recursion over strictly positive
inductive datatypes. The syntax of terms ($\vt$) and binders ($\vb$)
in this language is as follows:

\DM{
\begin{array}{rll@{\hg}|rll}
\vt ::= & \Type_i & (\mbox{type universes}) &
\hg\vb ::= & \lam{\vx}{\vt} & (\mbox{abstraction}) \\

 \mid  & \vx & (\mbox{variable}) &
 \mid & \LET\:\vx\defq\vt\Hab\vt & (\mbox{let binding}) \\

 \mid   & \vb\SC\:\vt & (\mbox{binding}) &
 \mid & \all{\vx}{\vt} & (\mbox{function space}) \\

 \mid   & \vt\:\vt & (\mbox{application})

\end{array}
}

We may also write the function space
\mbox{$\all{\vx}{\vS}\SC\vT$} as \mbox{$\fbind{\vx}{\vS}{\vT}$}, or
abbreviate it to \mbox{$\vS\to\vT$} if $\vx$ is not free in
$\vT$. This is both for readability and a notation more consistent
with traditional functional programming languages.

Universe levels on types (e.g. $\Type_0$ for values, $\Type_1$ for
types, etc.) may be left implicit and inferred
by the machine as in~\cite{implicit-pollack}.
Contexts ($\Gamma$) are collections of binders.

%% defined inductively; the empty context
%% is valid, as is a context extended with a $\lambda$, $\forall$ or
%% $\LET$ binding:

%% \DM{
%% \Axiom{\proves\RW{valid}}
%% \hg
%% \Rule{\Gamma\proves\RW{valid}}
%% {\Gamma;\vb\proves\RW{valid}}
%% }

The typing rules are given below. These depend on a conversion
relation \mbox{$\Gamma\proves\vx\conv\vy$}, which holds if and only if
$\vx$ and $\vy$ have a common reduct. This requires the typechecker to
perform some evaluation --- in order to find a common redex, types are
reduced to normal form --- so it is
important for decidability of typechecking that the language is
strongly normalising.

\DM{\begin{array}{c}
\Rule{\Gamma\proves\RW{valid}}
{\Gamma\vdash\Type_n\Hab\Type_{n+1}}\hspace*{0.1in}\mathsf{Type}
\\
\Rule{(\lam{\vx}{\vS})\in\Gamma}
{\Gamma\vdash\vx\Hab\vS}\hspace*{0.1in}\mathsf{Var}_1
\hg
\Rule{(\all{\vx}{\vS})\in\Gamma}
{\Gamma\vdash\vx\Hab\vS}\hspace*{0.1in}\mathsf{Var}_2
\hg
\Rule{(\LET\:\vx\Hab\vS\defq\vs)\in\Gamma}
{\Gamma\vdash\vx\Hab\vS}\hspace*{0.1in}\mathsf{Val}
\\
\Rule{\Gamma\vdash\vf\Hab\fbind{\vx}{\vS}{\vT}\hg\Gamma\vdash\vs\Hab\vS}
{\Gamma\vdash\vf\:\vs\Hab\vT[\vs/\vx]} % \LET\:\vx\Hab\vS\:\defq\:\vs\:\IN\:\vT}
\hspace*{0.1in}\mathsf{App}
\\

\Rule{\Gamma;\lam{\vx}{\vS}\vdash\ve\Hab\vT\hg\Gamma\proves\fbind{\vx}{\vS}{\vT}\Hab\Type_n}
{\Gamma\vdash\lam{\vx}{\vS}.\ve\Hab\fbind{\vx}{\vS}{\vT}}\hspace*{0.1in}\mathsf{Lam}
\\
\Rule{\Gamma;\all{\vx}{\vS}\vdash\vT\Hab\Type_n\hg\Gamma\vdash\vS\Hab\Type_n}
{\Gamma\vdash\fbind{\vx}{\vS}{\vT}\Hab\Type_n}\hspace*{0.1in}\mathsf{Forall}
\\

\Rule{\begin{array}{c}\Gamma\proves\ve_1\Hab\vS\hg
      \Gamma;\LET\:\vx\defq\ve_1\Hab\vS\proves\ve_2\Hab\vT\\
      \Gamma\proves\vS\Hab\Type_n\hg
      \Gamma;\LET\:\vx\defq\ve_1\Hab\vS\proves\vT\Hab\Type_n\end{array}
      }
{\Gamma\vdash\LET\:\vx\defq\ve_1\Hab\vS\SC\:\ve_2\Hab
   \vT[\ve_1/\vx]}   
%\Let\:\vx\Hab\vS\defq\ve_1\:\IN\:\vT}
\hspace*{0.1in}\mathsf{Let}
\\

\Rule{\Gamma\proves\vx\Hab\vA\hg\Gamma\proves\vA'\Hab\Type_n\hg
      \Gamma\proves\vA\converts\vA'}
     {\Gamma\proves\vx\Hab\vA'}
\hspace*{0.1in}\mathsf{Conv}
\end{array}
}
%{Typing rules for $\source$}
%{typerules}

\subsection{Inductive Families}

\label{indfamilies}

Inductive families \cite{dybjer94} are a form of simultaneously
defined collection of algebraic data types (such as Haskell
\texttt{data} declarations) which can be parametrised over
\remph{values} as well as types.  An inductive family is declared 
in a similar style to a Haskell GADT declaration~\cite{gadts}
as
follows, using the de Bruijn telescope notation, $\tx$, to indicate a
sequence of zero or more $\vx$:

\DM{
\AR{
\Data\:\TC{T}\:(\tx\Hab\ttt)\Hab\vt\hg\Where\hg
\DC{c}_1\Hab\vt\:\mid\:\ldots\:\mid\:\DC{c}_n\Hab\vt
}
}

Constructors may take recursive arguments in the family
$\TC{T}$. Furthermore these arguments may be indexed by another type,
as long it does not involve $\TC{T}$ --- this restriction is known as
\demph{strict positivity} and ensures that recursive arguments of a
constructor are structurally smaller than the value itself.

The Peano style natural numbers can be declared as follows:

\DM{
\Data\:\Nat\Hab\Type\hg\Where\hg\Z\Hab\Nat\:\mid\:\suc\Hab\fbind{\vk}{\Nat}{\Nat}
}

A data type may have zero or more parameters (which are invariant
across a structure) and a number of indices, given by the type. For
example, a list is parametrised over its element type:

\DM{
\AR{
\Data\:\List\:(\vA\Hab\Type)\Hab\Type\hg\Where
\ARd{
& \nil\Hab\List\:\vA\\
\mid & \cons\Hab\fbind{\vx}{\vA}{\fbind{\vxs}{\List\:\vA}{\List\:\vA}}
}
}
}

$\source$ is a dependently typed calculus, meaning that types can be
parametrised over values. Using this, we can declare the type of
vectors (lists with length), where the empty list is statically known
to have length zero, and the non empty list is statically known to
have a non zero length. $\Vect$ is parametrised over its element type,
like $\List$, but \remph{indexed} over its length.

\DM{
\AR{
\Data\:\Vect\:(\vA\Hab\Type)\Hab\Nat\to\Type\hg\Where \\
\hg\hg\ARd{
& \Vnil\Hab\Vect\:\vA\:\Z\\
\mid & \Vcons\Hab\fbind{\vk}{\Nat}{
\fbind{\vx}{\vA}{\fbind{\vxs}{\Vect\:\vA\:\vk}{\Vect\:\vA\:(\suc\:\vk)}}
}
}
}
}

\subsection{Elimination Rules}

\label{elimops}

When we declare an inductive family $\dD$, we give the constructors
which explain how to build objects in that family. Along with this,
\Ivor{} generates an \demph{elimination operator} $\delim$ (the
type of which we call the \demph{elimination rule}) and corresponding
reductions. The operator implements the reduction and recursion
behaviour of terms in the family --- it is effectively a fold
operator.  The method for constructing elimination operators
automatically is well documented, in particular
by~\cite{dybjer94,luo94,mcbride-thesis}.  For $\Vect$, \Ivor{} generates the
following operator:

\DM{
\AR{
\begin{array}{ll}
\vectelim\Hab & \fbind{\vA}{\Type}{\fbind{\vn}{\Nat}{
                \fbind{\vv}{\Vect\:\vA\:\vn}{}}} \\
              & \fbind{\motive}{
  \fbind{\vn}{\Nat}{\fbind{\vv}{\Vect\:\vA\:\vn}{\Type}}}{} 
   \\
              & \fbind{\meth{\Vnil}}{\motive\:\Z\:(\Vnil\:\vA)}{} \\
              & (\meth{\Vcons}\Hab
  \AR{
  \fbind{\vk}{\Nat}{\fbind{\vx}{\vA}{\fbind{\vxs}{\Vect\:\vA\:\vk}{}}} \\
  \fbind{\VV{ih}}{\motive\:\vk\:\vxs}{\motive\:(\suc\:\vk)\:
        (\Vcons\:\vA\:\vk\:\vx\:\vxs)})\to} \\
              & \motive\:\vn\:\vv 
\end{array}
\\
\PA{\A\A\A\A\A\A}{
& \vectelim & \vA & \Z & (\Vnil\:\vA) & \motive & \meth{\Vnil} & \meth{\Vcons} &
      \IRet{\meth{\Vnil}} \\
& \vectelim & \vA & (\suc\:\vk) & (\Vcons\:\vA\:\vk\:\vx\:\vxs) & \motive &
      \meth{\Vnil} & \meth{\Vcons} & \\
& & & \IMRet{6}{\meth{\Vcons}\:\vk\:\vx\:\vxs\:(\vectelim\:\vA\:\vk\:\vxs\:\motive\:\meth{\Vnil}\:\meth{\Vcons})} \\
}
}
}

The arguments to the elimination operator are the \demph{parameters}
and \demph{indices} ($\vA$ and $\vn$ here), the \demph{target} (the
object being eliminated; $\vv$ here), the \demph{motive} (a function
which computes the return type of the elimination; $\vP$ here) and the
\demph{methods} (which describe how to achieve the motive for each
constructor form).  Note the distinction between parameters and
indices --- the parameter $\vA$ is invariant across the structure so
is not passed to the methods as an argument, but $\vn$ does vary, so
is passed. A more detailed explanation of this distinction can be
found in~\cite{luo94,brady-thesis}.

A case analysis operator $\dcase$, is obtained similarly, but without
the induction hypotheses. These operators are the only means to
analyse a data structure and the only operators which can make
recursive calls. This, along with the restriction that data types must
be strictly positive, ensures that evaluation always terminates.

\subsection{The Development Calculus}

\label{sec:devcalc}

For developing terms interactively, the type theory needs to support
\remph{incomplete} terms, and a method for term construction. We
extend $\source$ with the concept of \demph{holes}, which stand for
the parts of constructions which have not yet been instantiated; this
largely follows McBride's \Oleg{} development
calculus~\cite{mcbride-thesis}.

The basic idea is to extend the syntax for binders with a \remph{hole}
binding and a \remph{guess} binding. The \remph{guess} binding is
similar to a $\LET$ binding, but without any computational force,
i.e. the bound names do not reduce:

\DM{
\vb ::= \ldots 
 \:\mid\: \hole{\vx}{\vt} \:\:(\mbox{hole binding}) \:\:
 \:\mid\: \guess{\vx}{\vt}{\vt} \:\:(\mbox{guess})
}

Using binders to represent holes as discussed in~\cite{mcbride-thesis}
is useful in a dependently typed setting since one value may determine
another. Attaching a ``guess'' to a binder ensures that instantiating one
such value also instantiates all of its dependencies. The typing rules for
binders ensure that no $?$ bindings leak into types, and are given
below.

\DM{
\AR{
\Rule{
\Gamma;\hole{\vx}{\vS}\proves\ve\Hab\vT
}
{
\Gamma\proves\hole{\vx}{\vS}\SC\ve\Hab\vT
}
\hspace*{0.1cm}\vx\not\in\vT
\hspace*{0.1in}\mathsf{Hole}
\hg
\Rule{
\Gamma;\guess{\vx}{\vS}{\ve_1}\proves\ve_2\Hab\vT
}
{
\Gamma\proves\guess{\vx}{\vS}{\ve_1}\SC\ve_2\Hab\vT
}
\hspace*{0.1cm}\vx\not\in\vT
\hspace*{0.1in}\mathsf{Guess}

}
}
%{Typing rules for $\source$ holes}
%{typerulesholes}

%% \subsection{Hole Manipulation}

%% \label{holeops}

%% Construction of terms through the \Ivor{} library relies on four basic
%% operations on holes: \demph{claim}, which introduces a new hole of a
%% given type; \demph{fill}, which attaches a guess to a hole;
%% \demph{abandon}, which removes a guess from a hole; and \demph{solve}
%% which finalises a guess by converting it to a $\LET$ binding,
%% providing that the guess is \remph{pure}, i.e. does not contain any
%% hole bindings or guesses.

%% \DM{
%% \begin{array}{l@{\hg}l}
%% \mbox{Claim} & 
%% \Rule{\Gamma\proves\ve\Hab\vT\hg
%% \Gamma\proves\vS\Hab\Type
%% }
%% {\Gamma\proves\hole{\vx}{\vS}\SC\ve\Hab\vT
%% }
%% \\
%% \mbox{Fill} & 
%% \Rule{\Gamma\proves\hole{\vx}{\vS}\SC\ve\Hab\vT\hg
%% \Gamma\proves\vv\Hab\vS}
%% {\Gamma\proves\guess{\vx}{\vS}{\vv}\SC\ve\Hab\vT}
%% \\
%% \mbox{Abandon} &
%% \Rule{\Gamma\proves\guess{\vx}{\vS}{\vv}\SC\ve\Hab\vT}
%% {\Gamma\proves\hole{\vx}{\vS}\SC\ve\Hab\vT}
%% \\
%% \mbox{Solve} &
%% \Rule{\Gamma\proves\guess{\vx}{\vS}{\vv}\SC\ve\Hab\vT}
%% {\Gamma\proves\LET\:\vx\Hab\vS\defq\:\vv\SC\ve\Hab\vT}
%% \hspace*{0.1cm}\vv\:\mbox{pure}

%% \end{array}
%% }




%\section{Basic Usage}

\subsection{Concrete Syntax}

\subsection{The Shell}

\subsection{The API}

\section{The \Ivor{} Library}

Given the basic operations defined in section \ref{holeops}, we can
create a library of tactics. In this section, I will introduce the
basic tactics available to the library user, along with the Haskell
interface for constructing and manipulating $\source$ terms. This
section includes only the most basic operations; the API is however
fully documented with the Haddock tool~\cite{haddock}, with
documentation available on the web\footnote{\url{http://www.dcs.st-and.ac.uk/~eb/Ivor/doc/}}.

\subsection{Definitions and Context}

The context, adding definitions, \texttt{ViewTerm} structure,
\texttt{IsTerm} class, text format, evaluation.

\subsection{Theorems}

\hdecl{
theorem :: (IsTerm a, Monad m) => Context -> Name -> a -> m Context
}

\subsection{Basic Tactics}

A tactic is an operation on a goal in the current system state; we
define a type synonym \hdecl{Tactic} for functions which operate as
tactics. Tactics modify system state and may fail, hence a tactic
function returns a monad:

\hdecl{type Tactic = forall m . Monad m => Goal -> Context -> m Context}

The proof state can be thought of as a $\source$ term, containing a
number of hole bindings. A tactic operates on one of these hole
bindings, specified by the \texttt{Goal} argument. This can be a named
binding, \texttt{goal :: Name -> Goal}, or the default goal
\texttt{defaultGoal::Name}. The default goal is the first goal
generated by the most recent tactic application.

\subsubsection{Hole Manipluations}

The four basic operations on holes, \demph{claim}, \demph{fill},
\demph{abandon} and \demph{solve} are given the following types:
\begin{verbatim}
claim :: IsTerm a => Name -> a -> Tactic
fill :: IsTerm a => a -> Tactic
abandon :: Tactic
solve :: Tactic
\end{verbatim}

The \texttt{claim} function takes a name and a type for the new hole,
and the \texttt{fill} function takes the guess to attach to the
specified hole. In addition, \texttt{fill} attempts to solve other
goals by unification.

It can be inconvenient to have to \texttt{solve} every goal after a
\texttt{fill} (although sometimes this level of control is
useful). For this reason, a tactic \texttt{keepSolving} is provided
which solves all goals with a hole-free guess attached,

\subsubsection{Introductions}

A basic operation on terms is to introduce $\lambda$ bindings into the
context. The \texttt{intro} and \texttt{introName} tactics operate on
a goal of the form $\fbind{\vx}{\vS}\to\vT$, introducing
$\lam{\vx}{\vS}$ into the context and updating the goal to
$\vT$. \texttt{introName} allows a user specified name choice,
otherwise \Ivor{} chooses the name.

\begin{verbatim}
intro :: Tactic
introName :: Name -> Tactic
\end{verbatim}

\subsubsection{Refinement}

The \texttt{refine} tactic solves a goal by an application of a
function to arguments. Refining attempts to solve a goal of type
$\vT$, when given a term $\vt\Hab\all{\tx}{\tS}\SC\vT$. The tactic
creates a subgoal for each argument $\vx_i$, attempting to solve it by
unfication.

\begin{verbatim}
refine :: IsTerm a => a -> Tactic
\end{verbatim}

For example, given a goal
\DM{
\hole{\vv}{\Vect\:\Nat\:(\suc\:\vn)}
}

Refining by $\Vcons$ creates subgoals for all four arguments, and
attaches a guess to $\vv$:
\DM{
\AR{
\hole{\vA}{\Type}\\
\hole{\vk}{\Nat}\\
\hole{\vx}{\vA}\\
\hole{\vxs}{\Vect\:\vA\:\vk}\\
\guess{\vv}{\Vect\:\Nat\:(\suc\:\vn)}{\Vcons\:\vA\:\vk\:\vx\:\vxs}
}
}

However, for $\Vcons\:\vA\:\vk\:\vx\:\vxs$ to have type
$\Vect\:\Nat\:(\suc\:\vn)$ requires that $\vA=\Nat$ and $\vk=\vn$.
Refinement unifies these, leaving the
following goals:
\DM{
\AR{
\hole{\vx}{\Nat}\\
\hole{\vxs}{\Vect\:\Nat\:\vn}\\
\guess{\vv}{\Vect\:\Nat\:(\suc\:\vn)}{\Vcons\:\Nat\:\vn\:\vx\:\vxs}
}
}

\subsubsection{Elimination}

Refinement solves goals by constructing new values; we may also solve
goals by deconstructing values in the context, using an elimination
operator as described in section \ref{elimops}. The \texttt{induction}
and \texttt{case} tactics apply the $\delim$ and $\dcase$ operators
respectively to the given target:

\begin{verbatim}
induction :: IsTerm a => a -> Tactic
cases :: IsTerm a => a -> Tactic
\end{verbatim}

These tactics proceed by refinement by the appropriate elimination
operator. The motive is calculated automatically, and is given by the
goal to be solved. Each tactic generates subgoals for each method.
A more general elimination tactic is \texttt{by}, which takes an
application of an elimination operator to a target.

\begin{verbatim}
by :: IsTerm a => a -> Tactic
\end{verbatim}

The type of the term given to \texttt{by} must be a function expecting
a motive and methods.

Example of induction?

\subsubsection{Rewriting}

\begin{verbatim}
replace :: (IsTerm a, IsTerm b, IsTerm c, IsTerm d) =>
    a -> b -> c -> d -> Bool -> Tactic
\end{verbatim}

\subsection{Tactic Combinators}

\subsubsection{Sequencing Tactics}

\begin{verbatim}
(>->) :: Tactic -> Tactic -> Tactic
(>=>) :: Tactic -> Tactic -> Tactic
(>+>) :: Tactic -> Tactic -> Tactic
\end{verbatim}

\begin{verbatim}
tacs :: Monad m => [Goal -> Context -> m Context] -> 
                   Goal -> Context -> m Context
\end{verbatim}

\subsubsection{Handling Failure}

\begin{verbatim}
try :: Tactic -> Tactic -> Tactic -> Tactic
\end{verbatim}

\subsection{The Shell}

\subsection{Extending and Embedding}


\section{Examples}

\subsection{A Propositional Logic Theorem Prover}

\label{example1}

Propositional logic is straightforward to model in dependent type
theory; here we show how \Ivor{} can be used to implement a theorem
prover for propositional logic. The full implementation is available
from \url{http://www.dcs.st-and.ac.uk/~eb/Ivor/}.  The language of
propositional logic is defined as follows, where $\vx$ stands for an
arbitrary free variable:

\DM{
\begin{array}{rll}
\vL ::= & \vx 
\mid \vL \land \vL
\mid \vL \lor \vL
\mid \vL \to \vL
\mid \neg\vL
\end{array}
}

\newcommand{\Tand}{\TC{And}}
\newcommand{\andintro}{\DC{and\_intro}}
\newcommand{\Tor}{\TC{Or}}
\newcommand{\orintrol}{\DC{inl}}
\newcommand{\orintror}{\DC{inr}}

There is a simple mapping from this language to dependent type theory
--- the $\land$ and $\lor$ connectives can be declared as inductive
families, where the automatically derived elimination rules give the
correct elimination behaviour, and the $\to$ connective follows the
same rules and the function arrow. Negation can be defined with the
empty type.

The $\land$ connective is declared as an inductive family, where an
instance of the family gives a proof of the connective. The $\andintro$
constructor builds a proof of $\vA\land\vB$, given proofs of $\vA$ and
$\vB$:

\DM{
\Data\:\Tand\:(\vA,\vB\Hab\Type)\Hab\Type\:=
\andintro\Hab\fbind{\va}{\vA}{\fbind{\vb}{\vB}{\Tand\:\vA\:\vB}}
}

Similarly, $\lor$ is declared as an inductive family; an instance of
$\vA\lor\vB$ is built either from a proof of $\vA$ ($\orintrol$) or a
proof of $\vB$ ($\orintror$):

\DM{
\AR{
\Data\:\Tor\:(\vA,\vB\Hab\Type)\Hab\Type\\
\hg=\:\orintrol\Hab\fbind{\va}{\vA}{\Tor\:\vA\:\vB}\\
\hg\mid\:\orintror\Hab\fbind{\vb}{\vB}{\Tor\:\vA\:\vB}\\
}
}

I will write $\interp{\ve}$ to denote the translate from an expression
$\ve\in\vL$ to an implementation in $\source$, in the implementation,
this is a parser from strings to \hdecl{ViewTerm}s:

\DM{
\AR{
\interp{\ve_1\land\ve_2}\:=\:\Tand\:\interp{\ve_1}\:\interp{\ve_2}\\
\interp{\ve_1\lor\ve_2}\:=\:\Tor\:\interp{\ve_1}\:\interp{\ve_2}\\
\interp{\ve_1\to\ve_2}\:=\:\interp{\ve_1}\to\interp{\ve_2}\\
}
}

To implement negation, we declare the empty type:

\DM{
\Data\:\False\Hab\Type\:=
}

Then $\interp{\neg\ve}\:=\:\interp{\ve}\to\False$. The automatically
derived elimination rule has the following type, showing that a value
of \remph{any} type can be created from a proof of the empty type:

\DM{
\Elim{\False}\Hab\fbind{\vx}{\False}{
\fbind{\motive}{\False\to\Type}{\motive\:\vx}}
}

In the implementation, we initialise the \hdecl{Context} with these
types (using \hdecl{addData}) and propositional variables
$\vA\ldots\vZ$ (using \hdecl{addAxiom}\footnote{This adds a name with
  a type but no definition to the context.}).

\mysubsubsection{Domain Specific Tactics}
Mostly, the implementation of a propositional logic theorem prover
consists of a parser and pretty printer for the language $\vL$, and a
top level loop for applying introduction and elimination
tactics. However, some domain
specific tactics are needed, in particular to deal with negation and
proof by contradiction. For example, the proof by contradiction tactic
is implemented as follows:

\texttt{contradiction :: Name -> Name -> Tactic}\\
\texttt{contradiction x y =}\\
\hspace*{0.5in}\texttt{claim $\vf$ $\False$ >-> induction $\vf$ >+>}\\
\hspace*{0.7in}\texttt{(try (fill ($\vx\:\vy$)) idTac (fill ($\vy\:\vx$)))}\\

This tactic takes the names of the two contradiction premises. One is
of type $\vA\to\False$ for some $\vA$, the other is of type
$\vA$. The tactic works by claiming there is a contradiction and
solving the goal by induction over that contradiction (which gives no
subgoals, since $\Elim{\False}$ has no methods). Finally, using
\texttt{>+>} to solve the next subgoal, it looks for a contradiction
by first applying $\vy$ to $\vx$ then, if that fails, applying $\vx$
to $\vy$.

\subsection{\Funl{}, a Functional Language with a Built-in Theorem Prover}

\label{example2}

Propositional logic is an example of a simple formal system which can
be embedded in a Haskell program using \Ivor{}; however, much more
complex languages can be implemented. \Funl{} is a simple functional
language, with primitive recursion over integers and higher order
functions. It is implemented on top of \Ivor{} as a framework for both
language representation and correctness proofs. By using \Ivor{}, it
is a small step from implementing the language to implementing a
theorem prover for showing properties of program in the language.

An implementation is available from
\url{http://www.dcs.st-and.ac.uk/~eb/Funl/}; in this section I will
sketch some of the important details of this implementation. Like the
propositional logic theorem prover, much of the detail is in the
parsing and pretty printing of terms and propositions.

\mysubsubsection{Building Terms}
Terms are parsed into the a data type \hdecl{Raw}; the name
\hdecl{Raw} reflects the fact that these are raw, untyped terms; note
in particular that \hdecl{Rec} is an operator for primitive recursion
on arbitrary types, like the $\delim$ operators in $\source$ --- it
would be fairly simple to write a first pass which translated
recursive calls into such an operator using techniques similar to
McBride and McKinna's labelled types~\cite{view-left}, which are
implemented in \Ivor{}. The representation is as follows:

\verb+data Raw = Var String | Lam String Ty Raw | App Raw Raw+\\
\verb+         | Num Int | Boolval Bool  | InfixOp Op Raw Raw+\\
\verb+         | If Raw Raw Raw | Rec Raw [Raw]+

Then the \hdecl{buildTerm} tactic traverses the
structure of the raw term, constructing a proof of the
theorem:

\hdecl{buildTerm :: Raw -> Tactic}

The definition is given in Appendix \ref{funlapp}.  \Ivor{} handles
the typechecking and any issues with renaming, using techniques from
\cite{not-a-number}; if there are any type errors in the \hdecl{Raw}
term, this tactic will fail (although some extra work is required to
produce readable error messages). By using \Ivor{} to handle
typechecking and evaluation, we are in no danger of evaluating a term
with type errors.


\mysubsubsection{Building Proofs}
We also define a language of propositions over terms in \Funl{}.
This uses propositional calculus, just like the theorem prover in
section \ref{example1}, but extended with equational reasoning. For
the equational reasoning, we use a library of equality proofs to
create tactics for applying commutativity and associativity of
addition and simplification of expressions.

A basic language of propositions with the obvious translation to
$\source$ is:

\verb+data Prop = Eq Raw Raw+\\
\verb+          | And Prop Prop | Or Prop Prop+\\
\verb+          | All String Ty Prop | FalseProp+

This allows equational reasoning over \Funl{} programs, quantification
over variables and conjunction and disjunction of propositions. A more
full featured prover may require relations other than \hdecl{Eq} or
even user defined relations.

\section{Related Work}

The ability to extend a theorem prover with user defined tactics has
its roots in Robin Milner's LCF~\cite{lcf-milner}. This introduced the
programming language ML to allow users to write tactics; we follow the
LCF approach in exposing the tactic engine as an API. However, unlike
other systems, we have not treated the theorem prover as an end in
itself, but intend to expose the technology to any Haskell application
which may need it.  The implementation of \Ivor{} is based on the
presentation of \Oleg{} in Conor McBride's
thesis~\cite{mcbride-thesis}; this technology also forms the basis for
the implementation of \Epigram{}~\cite{view-left}. The core language
of \Epigram{}~\cite{epireloaded} is similar to $\source$, with
extensions for observational equality. \Ivor{} uses implementation
techniques for \cite{not-a-number} for dealing with variables and
renaming, using de Bruijn indices.

Other theorem provers such as \Coq{}~\cite{coq-manual},
Agda~\cite{agda} and Isabell~\cite{isabelle} have varying degrees of
extensibility.  \Ivor{}'s interface design largely follows that of
\Coq{}. \Coq{} includes a high level domain specific language for
combining tactics and creating new tactics, along the lines of the
tactic combinators presented in section \ref{combinators}. This
language is ideal for many purposes, such as our \hdecl{contradiction}
tactic, but more complex examples such as \hdecl{buildTerm} would
require extending \Coq{} itself.

Isabelle~\cite{isabelle} is a generic theorem prover, in that it
includes a large body of object logics and a meta-language for
defining new logics. It includes a typed, extensible tactic language,
and can be called from ML programs, but unlike \Ivor{} is not based on
a dependent type theory.

The implementation of \Funl{} allows a theorem prover to be attached
to the language in a straightforward way, using \Ivor{}'s tactics
directly. This would be a possible method of attaching a theorem
prover to a more full featured programming language such as the
Sparkle~\cite{sparkle} prover for Clean~\cite{clean}.

\section{Conclusions}

Didn't talk about primitive types, general recursion or all sorts of
other bits and pieces. See web site.

Easy to make tools with proof gadgets attached, or verified
implementations of programming languages.

\subsection{Further Work}

Development driven by our research into Hume and dependent types.
We'll need some more tactics, e.g. Elimination with a Motive
\cite{elim-motive}. Perhaps a compiler, and optimisations from
\cite{brady-thesis}. An implementation of
coinduction~\cite{coinductive} would be nice, currently it's faked by
recursive functions which don't reduce at the type level, but a proper
implementation, perhaps with a syntactic criterion for correctness
would be better.


\section*{Acknowledgements}

This work is generously supported by EPSRC grant EP/C001346/1. 
%My thanks to
%\ldots

\bibliographystyle{abbrv}
\begin{small}
\bibliography{../bib/literature.bib}

\appendix

\section{Haskell Code}

This appendix contains some of the more important definitions from the theorem
prover and functional language implementation. The complete code for
both examples is available from
\url{http://www.dcs.st-and.ac.uk/~eb/Ivor}; the code presented here
illustrates the building of complex tactics with \Ivor{}.

\subsection{Propositional Logic}

Two domain specific tactics are needed; firstly to prove a negation
$\neg\vA$, we assume $\vA$ and attempt to prove $\bot$. This is
achieved with a \hdecl{negate} tactic; \hdecl{getGoal} is used to get
the type of the current goal, and \hdecl{view} is used to convert the
internal representation into a pattern matchable representation.

\begin{verbatim}
> negate :: Tactic
> negate g ctxt = do 
>    (_,gtype) <- getGoal ctxt g
>    assumption <- negateTerm (view gtype)
>    ctxt <- equiv (Forall (name "neg") assumption false) g ctxt
>    intro g ctxt
\end{verbatim}

\begin{verbatim}
> negateTerm :: Monad m => ViewTerm -> m ViewTerm
> negateTerm (App (Name _ n) arg) | n == (name "not") = return arg
> negateTerm _ = fail "Not a negation"
\end{verbatim}

Secondly, we need a tactic to prove a contradiction as discussed in
section \ref{example1}:

\begin{verbatim}
> contradiction :: String -> String -> Tactic
> contradiction x y = claim (name "false") "False" >+>
>                     induction "false" >+>
>                     (try (fill $ x ++ " " ++ y)
>                           idTac
>                           (fill $ y ++ " " ++ x))
\end{verbatim}

\subsection{\Funl{}}

\label{funlapp}

When building a function definition, we prove a \hdecl{theorem} of the
appropriate type. Then the \hdecl{buildTerm} tactic traverses the
structure of the raw term, constructing a proof of the
theorem. \Ivor{} handles the typechecking; if there are any type
errors, this tactic will fail.

\begin{verbatim}
> buildTerm :: Raw -> Tactic
> buildTerm (Var x) = refine x
> buildTerm (Lam x ty sc) = introName (name x) >+> buildTerm sc
> buildTerm (Language.App f a) = buildTerm f >+> buildTerm a
> buildTerm (Num x) = fill (mkNat x)
> buildTerm (If a t e) = 
>     cases (mkTerm a) >+> buildTerm t >+> buildTerm e
> buildTerm (Rec t alts) =
>     induction (mkTerm t) >+> tacs (map buildTerm alts)
> buildTerm (InfixOp Plus x y) = 
>     refine "plus" >+> buildTerm x >+> buildTerm y
> buildTerm (InfixOp Times x y) = ...
\end{verbatim}

A helper function, \hdecl{mkTerm}, is used to translate simple
expressions into \hdecl{ViewTerm}s:

\begin{verbatim}
> mkTerm :: Raw -> ViewTerm
> mkTerm (Var x) = (Name Unknown (name x))
> mkTerm (Lam x ty sc) = Lambda (name x) (mkType ty) (mkTerm sc)
> mkTerm (Apply f a) = App (mkTerm f) (mkTerm a)
> mkTerm (Num x) = mkNat x
> mkTerm (InfixOp Plus x y) = 
>     App (App (Name Free (name "plus")) (mkTerm x)) (mkTerm y)
> mkTerm (InfixOp Times x y) = ...
> mkTerm _ = error "Term is too complicated in this context"
\end{verbatim}

\end{small}
\end{document}

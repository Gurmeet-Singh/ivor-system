\section{Related Work}

The ability to extend a theorem prover with user defined tactics has
its roots in Robin Milner's LCF~\cite{lcf-milner}. This introduced the
programming language ML to allow users to write tactics; we follow the
LCF approach in exposing the tactic engine as an API. However, unlike
other systems, we have not treated the theorem prover as an end in
itself, but intend to expose the technology to any Haskell application
which may need it.  The implementation of \Ivor{} is based on the
presentation of \Oleg{} in Conor McBride's
thesis~\cite{mcbride-thesis}. We use implementation
techniques from \cite{not-a-number} for dealing with variables and
renaming.

The core language of \Epigram{}~\cite{view-left,epireloaded} is
similar to $\source$, with extensions for observational
equality. \Epigram{} is a dependently typed functional programming
language, where types can be predicated on arbitrary values so that
types can be read as precise specifications.
Another recent language which shares the aim of begin theorem proving
technology closer to programers is Sheard's
$\Omega$mega~\cite{sheard-langfuture}. While \Ivor{} emphasises
interactive theorem proving, $\Omega$mega emphasises programming but
nevertheless allows more precise types to be given to programs through
Generalised Algebraic Data Types~\cite{gadts} and extensible
kinds. 

Other theorem provers such as \Coq{}~\cite{coq-manual},
\Agda{}~\cite{agda} and Isabelle~\cite{isabelle} have varying degrees
of extensibility. The interface design largely follows that of
\Coq{}. \Coq{} includes a high level domain specific language for
combining tactics and creating new tactics, along the lines of the
tactic combinators presented in section \ref{combinators}. This
language is ideal for many purposes, such as our \hdecl{contradiction}
tactic, but more complex examples such as \hdecl{buildTerm} would
require extending \Coq{} itself. One advantage of using a
DSEL~\cite{hudak-edsl} as provided by \Ivor{} is that it gives complete
flexibility in the construction of tactics, and allows a close
relationship between the tactics and the Haskell-implemented
structures on which they operate (e.g. \hdecl{Raw}). In future, it may
be worth exploring automatic translation between \Ivor{} and other
theorem provers.

Isabelle~\cite{isabelle} is a
generic theorem prover, in that it includes a large body of object
logics and a meta-language for defining new logics. It includes a
typed, extensible tactic language, and can be called from ML programs,
but unlike \Ivor{} is not based on a dependent type theory.
There is therefore no \remph{proof term} associated with an Isabelle
proof --- the proof term gives a derivation tree for the
proof, allowing easy and independent rechecking without referring to
the tactics used to build the proof. 

The implementation of \Funl{} allows a theorem prover to be attached
to the language in a straightforward way, using \Ivor{}'s tactics
directly. This would be a possible method of attaching a theorem
prover to a more full featured programming language such as the
Sparkle~\cite{sparkle} prover for Clean~\cite{clean}. Implementing a
full language in this way would require some extra work to deal
with general recursion and partial definitions (in particular, dealing
with $\perp$ as a possible value), but the general method remains the same.

\section{Conclusions and Further Work}

We have seen an overview of the \Ivor{} library, including basic
tactics for building proofs similar to the tactics available in other
proof assistants such as \Coq{}. By exposing the tactic API and
providing an interface for term construction and evaluation, we are
able to embed theorem proving technology in a Haskell
application. This in itself is not a new idea, having first been seen
as far back as the LCF~\cite{lcf-milner} prover --- however, the
theorem proving technology is not an end in itself, but a
mechanism for constructing domain specific tools such as the
propositional logic theorem prover in section \ref{example1} and the
programming language with built in equational reasoning support in
section \ref{example2}.

The library includes several features we have not been able to discuss
here. Dependently typed pattern matching 

There is experimental support for multi-stage programming with
dependent types, exploited in~\cite{dtpmsp-gpce}.  The term language
can be extended with primitive types and operations, e.g. integers and
strings with associated arithmetic and string manipulation
operators. Such features would be essential in a representation of a
real programming language. In this paper, we have stated that
$\source$ is strongly normalising, with no general recursion allowed,
but again in the representation of a real programming language general
recursion may be desirable --- however, this means that correctness
proofs can no longer be total. The library can optionally allow
general recursive definitions, but such definitions cannot be reduced
by the typechecker. Finally, a command driven interface is available,
which can be accessed as a Haskell API or used from a command line
driver program, and allows user directed proof scripts in the style of
other proof assistants. These and other features are fully documented
on the web
site\footnote{\url{http://www.cs.st-andrews.ac.uk/~eb/Ivor/}}.

%% \subsection{Further Work}

Development of the library has been driven by the requirements of
our research into Hume~\cite{Hume-GPCE}, a resource aware functional
language. We are investigating the use of dependent types in
representing and verifying resource bounded functional
programs~\cite{dt-framework}. For this, automatic generation of
injectivity and disjointness lemmas for constructors will be
essential~\cite{concon}, as well as an elimination with a
motive~\cite{elim-motive} tactic. Future versions will include
optimisations from \cite{brady-thesis} and some support for compiling
$\source$ terms; this would not only improve the efficiency of the
library (and in particular its use for evaluating certified code)
but also facilitate the use of \Ivor{} in a real language
implementation. Finally, an implementation of coinductive
types~\cite{coinductive} is likely to be very useful; currently it can
be achieved by implementing recursive functions which do not reduce at
the type level, but a complete implementation with criteria for
checking productivity would be valuable for modelling streams in Hume.

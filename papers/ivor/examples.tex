\section{Examples}

In this section we show two examples of embedding \Ivor{} in a Haskell
program. The first shows an embedding of a simple theorem prover for
propositional logic. The second example extends this theorem prover by
using the same logic as a basis for showing properties of a functional
language.

\subsection{A Propositional Logic Theorem Prover}

\label{example1}

Propositional logic is straightforward to model in dependent type
theory; here we show how \Ivor{} can be used to implement a theorem
prover for propositional logic. The full implementation is available
from \url{http://www.cs.st-andrews.ac.uk/~eb/Ivor/}.  The language of
propositional logic is defined as follows, where $\vx$ stands for an
arbitrary free variable:

\DM{
\begin{array}{rll}
\vL ::= & \vx 
\mid \vL \land \vL
\mid \vL \lor \vL
\mid \vL \to \vL
\mid \neg\vL
\end{array}
}

\newcommand{\Tand}{\TC{And}}
\newcommand{\andintro}{\DC{and\_intro}}
\newcommand{\Tor}{\TC{Or}}
\newcommand{\orintrol}{\DC{inl}}
\newcommand{\orintror}{\DC{inr}}

There is a simple mapping from this language to dependent type theory
--- the $\land$ and $\lor$ connectives can be declared as inductive
families, where the automatically derived elimination rules give the
correct elimination behaviour, and the $\to$ connective follows the
same rules as the function arrow. Negation can be defined with the
empty type.

The $\land$ connective is declared as an inductive family, where an
instance of the family gives a proof of the connective. The $\andintro$
constructor builds a proof of $\vA\land\vB$, given a proof of $\vA$ and
a proof of $\vB$:

\DM{
\AR{
\Data\:\Tand\:(\vA,\vB\Hab\Type)\Hab\Type\hg\Where\\
\hg\hg
\andintro\Hab\fbind{\va}{\vA}{\fbind{\vb}{\vB}{\Tand\:\vA\:\vB}}
}
}

Similarly, $\lor$ is declared as an inductive family; an instance of
$\vA\lor\vB$ is built either from a proof of $\vA$ ($\orintrol$) or a
proof of $\vB$ ($\orintror$):

\DM{
\AR{
\Data\:\Tor\:(\vA,\vB\Hab\Type)\Hab\Type\hg\Where\\
\hg\hg\ARd{
& \orintrol\Hab\fbind{\va}{\vA}{\Tor\:\vA\:\vB}\\
\mid & \orintror\Hab\fbind{\vb}{\vB}{\Tor\:\vA\:\vB}
}
}
}

I will write $\interp{\ve}$ to denote the translate from an expression
$\ve\in\vL$ to an implementation in $\source$; in the implementation,
this is a parser from strings to \hdecl{ViewTerm}s:

\DM{
\AR{
\interp{\ve_1\land\ve_2}\:=\:\Tand\:\interp{\ve_1}\:\interp{\ve_2}\\
\interp{\ve_1\lor\ve_2}\:=\:\Tor\:\interp{\ve_1}\:\interp{\ve_2}\\
\interp{\ve_1\to\ve_2}\:=\:\interp{\ve_1}\to\interp{\ve_2}\\
}
}

To implement negation, we declare the empty type:

\DM{
\Data\:\False\Hab\Type\hg\Where
}

Then $\interp{\neg\ve}\:=\:\interp{\ve}\to\False$. The automatically
derived elimination rule has the following type, showing that a value
of \remph{any} type can be created from a proof of the empty type:

\DM{
\Elim{\False}\Hab\fbind{\vx}{\False}{
\fbind{\motive}{\False\to\Type}{\motive\:\vx}}
}

In the implementation, we initialise the \hdecl{Context} with these
types (using \hdecl{addData}) and propositional variables
$\vA\ldots\vZ$ (using \hdecl{addAxiom}\footnote{This adds a name with
  a type but no definition to the context.}).

\subsubsection{Domain Specific Tactics}
Mostly, the implementation of a propositional logic theorem prover
consists of a parser and pretty printer for the language $\vL$, and a
top level loop for applying introduction and elimination
tactics. However, some domain
specific tactics are needed, in particular to deal with negation and
proof by contradiction. 

To prove a negation $\neg\vA$, we assume $\vA$ and attempt to prove
$\False$. This is achieved with an \hdecl{assumeFalse} tactic which
assumes the negation of the goal. Negation is defined with a function
$\FN{not}$; the \hdecl{assumeFalse} tactic then unfolds this name so
that a goal (in $\source$ syntax) $\FN{not}\:\vA$ is transformed to
$\vA\:\to\:\False$, then $\vA$ can be introduced.

\begin{verbatim}
assumeFalse :: Tactic
assumeFalse = unfold (name "not") >+> intro 
\end{verbatim}

The proof by contradiction tactic
is implemented as follows:
\\

\begin{verbatim}
contradiction :: String -> String -> Tactic
contradiction x y = claim (name "false") "False" >+>
                    induction "false" >+>
                    (try (fill $ x ++ " " ++ y)
                          idTac
                          (fill $ y ++ " " ++ x))
\end{verbatim}

This tactic takes the names of the two contradiction premises. One is
of type $\vA\to\False$ for some $\vA$, the other is of type $\vA$. The
tactic works by claiming there is a contradiction and solving the goal
by induction over that assumed contradiction (which gives no subgoals,
since $\Elim{\False}$ has no methods). Finally, using \texttt{>+>} to
solve the next subgoal (and discharge the assumption of the
contradiction), it looks for a value of type $\False$ by first
applying $\vy$ to $\vx$ then, if that fails, applying $\vx$ to $\vy$.

\subsection{\Funl{}, a Functional Language with a Built-in Theorem Prover}

\label{example2}

Propositional logic is an example of a simple formal system which can
be embedded in a Haskell program using \Ivor{}; however, more complex
languages can be implemented. \Funl{} is a simple functional language,
with primitive recursion over integers and higher order functions. It
is implemented on top of \Ivor{} as a framework for both language
representation and correctness proofs in that language. By using the
same framework for both, it is a small step from implementing the
language to implementing a theorem prover for showing properties of
programs in the language, making use of the theorem prover developed
in sec. \ref{example1}.
An implementation is available from
\url{http://www.cs.st-andrews.ac.uk/~eb/Funl/}; in this section I will
sketch some of the important details of this implementation. Like the
propositional logic theorem prover, much of the detail is in the
parsing and pretty printing of terms and propositions.

\subsubsection{Programs and Properties}

The \Funl{} language allows programs and statements of the properties
those programs should satisfy to be expressed within the same input file.
Functions are defined as in the following examples, using \hdecl{rec}
to mark a primitive recursive definition:

\begin{verbatim}
fac : Int -> Int =
     lam x . rec x 1 (lam k. lam recv. (k+1)*recv);
myplus : Int -> Int -> Int =
     lam x. lam y. rec x y (lam k. lam recv. 1+recv);
\end{verbatim}

The \hdecl{myplus} function above defines addition by primitive
recursion over its input. To show that this really is a definition of
addition, we may wish to show that it satisfies some appropriate
properties of addition. In the \Funl{} syntax, we declare the
properties we wish to show as follows:

\begin{verbatim}
myplusn0 proves 
     forall x:Int. myplus x 0 = x;
myplusnm1 proves 
     forall x:Int. forall y:Int. myplus x (1+y) = 1+(myplus x y);
myplus_commutes proves 
     forall x:Int. forall y:Int. myplus x y = myplus y x;
\end{verbatim}

On compiling a program, the compiler requires that proofs are provided
for the stated properties. Once built, the proofs can be saved as
proof terms, so that properties need only be proved once.

\subsubsection{Building Terms}
Terms are parsed into a data type \hdecl{Raw}; the name
\hdecl{Raw} reflects the fact that these are raw, untyped terms; note
in particular that \hdecl{Rec} is an operator for primitive recursion
on arbitrary types, like the $\delim$ operators in $\source$ --- it
would be fairly simple to write a first pass which translated
recursive calls into such an operator using techniques similar to
McBride and McKinna's labelled types~\cite{view-left}, which are
implemented in \Ivor{}. Using this, we could easily extend the
language with more primitive types (e.g. lists) or even user defined
data types. The representation is as follows:

\begin{verbatim}
data Raw = Var String | Lam String Ty Raw | App Raw Raw
         | Num Int | Boolval Bool  | InfixOp Op Raw Raw
         | If Raw Raw Raw | Rec Raw [Raw]
\end{verbatim}

Building a \Funl{} function consists of creating a \hdecl{theorem} 
with a goal representing the function's type, then using the
\hdecl{buildTerm} tactic to traverse the
structure of the raw term, constructing a proof of the
theorem --- note especially that \texttt{rec} translates into an
application of the appropriate elimination rule via the
\texttt{induction} tactic:

\begin{verbatim}
buildTerm :: Raw -> Tactic
buildTerm (Var x) = refine x
buildTerm (Lam x ty sc) = introName (name x) >+> buildTerm sc
buildTerm (Language.App f a) = buildTerm f >+> buildTerm a
buildTerm (Num x) = fill (mkNat x)
buildTerm (If a t e) = 
    cases (mkTerm a) >+> buildTerm t >+> buildTerm e
buildTerm (Rec t alts) =
    induction (mkTerm t) >+> tacs (map buildTerm alts)
buildTerm (InfixOp Plus x y) = 
    refine "plus" >+> buildTerm x >+> buildTerm y
buildTerm (InfixOp Times x y) = ...
\end{verbatim}

A helper function, \hdecl{mkTerm}, is used to translate simple
expressions into \hdecl{ViewTerm}s. This is used for the scrutinees of
\hdecl{if} and \hdecl{rec} expressions, although if more complex
expressions are desired here, it would be possible to use
\hdecl{buildTerm} instead.

\begin{verbatim}
mkTerm :: Raw -> ViewTerm
mkTerm (Var x) = (Name Unknown (name x))
mkTerm (Lam x ty sc) = Lambda (name x) (mkType ty) (mkTerm sc)
mkTerm (Apply f a) = App (mkTerm f) (mkTerm a)
mkTerm (Num x) = mkNat x
mkTerm (InfixOp Plus x y) = 
    App (App (Name Free (name "plus")) (mkTerm x)) (mkTerm y)
mkTerm (InfixOp Times x y) = ...
\end{verbatim}

\Ivor{} handles
the typechecking and any issues with renaming, using techniques from
\cite{not-a-number}; if there are any type errors in the \hdecl{Raw}
term, this tactic will fail (although some extra work is required to
produce readable error messages). By using \Ivor{} to handle
typechecking and evaluation, we are in no danger of constructing or 
evaluating an ill-typed term.


\subsubsection{Building Proofs}
We also define a language of propositions over terms in \Funl{}.
This uses propositional logic, just like the theorem prover in
section \ref{example1}, but extended with equational reasoning. For
the equational reasoning, we use a library of equality proofs to
create tactics for applying commutativity and associativity of
addition and simplification of expressions.

A basic language of propositions with the obvious translation to
$\source$ is:

\begin{verbatim}
data Prop = Eq Raw Raw
          | And Prop Prop | Or Prop Prop
          | All String Ty Prop | FalseProp
\end{verbatim}

This allows equational reasoning over \Funl{} programs, quantification
over variables and conjunction and disjunction of propositions. A more
full featured prover may require relations other than \hdecl{Eq} or
even user defined relations.
\section{Examples}

\subsection{A Propositional Logic Theorem Prover}

\label{example1}

Propositional logic is straightforward to model in dependent type
theory; here we show how \Ivor{} can be used to implement a theorem
prover for propositional logic. The language of propositional logic is
defined as follows, where $\vx$ stands for an arbitrary free variable:

\DM{
\begin{array}{rll}
\vL ::= & \vx 
\mid \vL \land \vL
\mid \vL \lor \vL
\mid \vL \to \vL
\mid \neg\vL
\end{array}
}

\newcommand{\Tand}{\TC{And}}
\newcommand{\andintro}{\DC{and\_intro}}
\newcommand{\Tor}{\TC{Or}}
\newcommand{\orintrol}{\DC{inl}}
\newcommand{\orintror}{\DC{inr}}

There is a simple mapping from this language to dependent type theory
--- the $\land$ and $\lor$ connectives can be declared as inductive
families, where the automatically derived elimination rules give the
correct elimination behaviour, and the $\to$ connective follows the
same rules and the function arrow. Negation can be defined with the
empty type.

The $\land$ connective is declared as an inductive family, where an
instance of the family gives a proof of the connective. The $\andintro$
constructor builds a proof of $\vA\land\vB$, given proofs of $\vA$ and
$\vB$:

\DM{
\Data\:\Tand\:(\vA,\vB\Hab\Type)\Hab\Type\:=
\andintro\Hab\fbind{\va}{\vA}{\fbind{\vb}{\vB}{\Tand\:\vA\:\vB}}
}

Similarly, $\lor$ is declared as an inductive family; an instance of
$\vA\lor\vB$ is built either from a proof of $\vA$ ($\orintrol$) or a
proof of $\vB$ ($\orintror$):

\DM{
\AR{
\Data\:\Tor\:(\vA,\vB\Hab\Type)\Hab\Type\\
\hg=\:\orintrol\Hab\fbind{\va}{\vA}{\Tor\:\vA\:\vB}\\
\hg\mid\:\orintror\Hab\fbind{\vb}{\vB}{\Tor\:\vA\:\vB}\\
}
}

I will write $\interp{\ve}$ to denote the translate from an expression
$\ve\in\vL$ to an implementation in $\source$, in the implementation,
this is a parser from strings to \hdecl{ViewTerm}s:

\DM{
\AR{
\interp{\ve_1\land\ve_2}\:=\:\Tand\:\interp{\ve_1}\:\interp{\ve_2}\\
\interp{\ve_1\lor\ve_2}\:=\:\Tor\:\interp{\ve_1}\:\interp{\ve_2}\\
\interp{\ve_1\to\ve_2}\:=\:\interp{\ve_1}\to\interp{\ve_2}\\
}
}

To implement negation, we declare the empty type:

\DM{
\Data\:\False\Hab\Type\:=
}

Then $\interp{\neg\ve}\:=\:\interp{\ve}\to\False$. The automatically
derived elimination rule has the following type, showing that a value
of \remph{any} type can be created from a proof of the empty type:

\DM{
\Elim{\False}\Hab\fbind{\vx}{\False}{
\fbind{\motive}{\False\to\Type}{\motive\:\vx}}
}

In the implementation, we initialise the \hdecl{Context} with these
types (using \hdecl{addData}) and propositional variables
$\vA\ldots\vZ$ (using \hdecl{addAxiom}\footnote{This adds a name with
  a type but no definition to the context.}).

\subsubsection{Domain Specific Tactics}

Mostly, the implementation of a propositional logic theorem prover
consists of a parser and pretty printer for the language $\vL$, and a
top level loop for applying introduction and elimination
tactics. However, some domain
specific tactics are needed, in particular to deal with negation and
proof by contradiction. For example, the proof by contradiction tactic
is implemented as follows:

\texttt{contradiction :: Name -> Name -> Tactic}\\
\texttt{contradiction x y =}\\
\hspace*{0.5in}\texttt{claim $\vf$ $\False$ >-> induction $\vf$ >+>}\\
\hspace*{0.7in}\texttt{(try (fill ($\vx\:\vy$)) idTac (fill ($\vy\:\vx$)))}\\

This tactic takes the names of the two contradiction premises. One is
of type $\vA\to\False$ for some $\vA$, the other is of type
$\vA$. The tactic works by claiming there is a contradiction and
solving the goal by induction over that contradiction (which gives no
subgoals, since $\Elim{\False}$ has no methods). Finally, using
\texttt{>+>} to solve the next subgoal, it looks for a contradiction
by first applying $\vy$ to $\vx$ then, if that fails, applying $\vx$
to $\vy$.

The full implementation is available from
\url{http://www.dcs.st-and.ac.uk/~eb/Ivor/}. 

\subsection{\Funl{}, a Functional Language with a Built-in Theorem Prover}

\label{example2}

Propositional logic is an example of a simple formal system which can
be embedded in a Haskell program using \Ivor{}; however, much more
complex languages can be implemented. \Funl{} is a simple functional
language, with primitive recursion over integers and higher order
functions. It is implemented on top of \Ivor{} as a framework for both
language representation and correctness proofs. By using \Ivor{}, it
is a small step from implementing the language to implementing a
theorem prover for showing properties of program in the language.

An implementation is available from
\url{http://www.dcs.st-and.ac.uk/~eb/Funl/}; in this section I will
sketch some of the important details of this implementation. Like the
propositional logic theorem prover, much of the detail is in the
parsing and pretty printing of terms and propositions.

\subsubsection{Building Terms}

Terms are parsed into the a data type \hdecl{Raw}; the name
\hdecl{Raw} reflects the fact that these are raw, untyped terms; note
in particular that \hdecl{Rec} is an operator for primitive recursion
on arbitrary types, like the $\delim$ operators in $\source$ --- it
would be fairly simple to write a first pass which translated
recursive calls into such an operator using techniques similar to
McBride and McKinna's labelled types~\cite{view-left}, which are
implemented in \Ivor{}. The representation is as follows:

\begin{verbatim}
data Raw = Var String | Lam String Ty Raw | App Raw Raw
         | Num Int | Boolval Bool  | InfixOp Op Raw Raw
         | If Raw Raw Raw | Rec Raw [Raw]
\end{verbatim}

Then the \hdecl{buildTerm} tactic traverses the
structure of the raw term, constructing a proof of the
theorem:

\begin{verbatim}
buildTerm :: Raw -> Tactic
\end{verbatim}

The definition is given in Appendix \ref{funlapp}.  \Ivor{} handles
the typechecking and any issues with renaming, using techniques from
\cite{not-a-number}; if there are any type errors in the \hdecl{Raw}
term, this tactic will fail (although some extra work is required to
produce readable error messages). By using \Ivor{} to handle
typechecking and evaluation, we are in no danger of evaluating a term
with type errors.


\subsubsection{Building Proofs}

We also define a language of propositions over terms in \Funl{}.
This uses propositional calculus, just like the theorem prover in
section \ref{example1}, but extended with equational reasoning. For
the equational reasoning, we use a library of equality proofs to
create tactics for applying commutativity and associativity of
addition and simplification of expressions.

A basic language of propositions can be defined as follows, with the
obvious translation into $\source$:

\begin{verbatim}
data Prop = Eq Raw Raw
          | And Prop Prop | Or Prop Prop
          | All String Ty Prop | FalseProp
\end{verbatim}

This allows equational reasoning over \Funl{} programs, quantification
over variables and conjunction and disjunction of propositions. A more
full featured prover may require relations other than \hdecl{Eq} or
even user defined relations.
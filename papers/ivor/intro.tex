\section{Introduction}

%\Ivor{} is a tactic-based theorem proving engine with a Haskell
%API. Unlike other systems such as \Coq{}~\cite{coq-manual} and
%Agda~\cite{agda}, the tactic engine is primarily intended to be used
%by programs, rather than a human operator. 

Type theory based theorem provers such as \Coq{}~\cite{coq-manual} and
\Agda{}~\cite{agda} have been used as tools for verification of programs
(e.g.~\cite{leroy-compiler,why-tool,mckinna-expr}), extraction of
correct programs from proofs (e.g.~\cite{extraction-coq})
and formal proofs of mathematical properties
(e.g.~\cite{fta,four-colour}).  However, these tools are designed with a
human operator in mind; the interface is textual which makes it
difficult for an external program to interact with them. 
In contrast, the \Ivor{} library is designed to provide an
implementation of dependent type theory (i.e. dependently typed
$\lambda$-calculus) and tactics for proof and
program development to a Haskell application programmer, via a stable,
well-documented and lightweight (as far as possible) API. The goal is
to allow: i) easy embedding of theorem proving tools in a Haskell
application; and ii) easy extension of the theorem prover with
\remph{domain specific} tactics, via a domain specific embedded
language (DSEL) for tactic construction.

%% \Coq{}
%% provides an extraction mechanism~\cite{extraction-coq} which generates
%% ML or Haskell code from a proof term, but this does not allow the easy
%% \remph{construction} of proof terms by an external tool. It is also
%% extensible to some extent, for example using a domain specific
%% language for creating user tactics, but the result is difficult to
%% embed in an external program.

%More
%recently, dependent types have been incorporated into programming
%languages such as Cayenne~\cite{cayenne-icfp}, DML~\cite{xi-thesis} and
%\Epigram{}~\cite{view-left,epigram-afp}.


%% have been used for several
%% large practical applications, including correctness proofs for a
%% compiler~\cite{leroy-compiler} and a computer assisted proof of the
%% four colour theorem~\cite{four-colour}. 

\subsection{Motivating Examples}

Many situations can benefit from a dependently typed proof and
programming framework accessible as a library from a Haskell program.
For each of these, by using an implementation of a well understood
type theory, we can be confident that the underlying framework is
sound. 

%% --- provided of course that the implementation itself is correct. 
%% Implementing
%% eliminates the need to prove that a language and proof system are consistent with
%% each other or that a special purpose proof framework is sound.

\begin{description}
\item[Programming Languages] 
Dependent type theory is a possible internal representation for a
functional programming language. 
%The core language of the Glasgow
%Haskell Compiler is \SystemF{}~\cite{core} --- dependent type theory
%generalises this by allowing types to be parametrised over values.
Correctness
properties of programs in purely functional languages can be proven by
equational reasoning, 
e.g. with Sparkle~\cite{sparkle} for the Clean language~\cite{clean}, or
Cover~\cite{cover} for translating Haskell into
\Agda{}~\cite{agda}. However these tools 
separate the language implementation from the theorem prover --- every
language feature must be translated into the theorem prover's
representation, and any time the language implementation is changed,
this translation must also be changed.
In section
\ref{example2}, we will see how \Ivor{} can be used to implement a
language with a built-in theorem prover, with a common representation
for both.

\item[Verified DSL Implementation] 

We have previously implementated a verified domain specific
language~\cite{dtpmsp-gpce} with \Ivor{}.  The abstract syntax tree of
a program is a dependent data structure, and the type system
guarantees that invariant properties of the program are maintained
during evaluation.  Using staging annotations~\cite{multi-taha}, such
an interpreter can be specialised to a translator. We are continuing
to explore these techniques in the context of resource aware
programming~\cite{dt-framework}.

\item[Formal Systems] 

A formal system can be modelled in dependent type theory, and
derivations within the system can be constructed and checked. 
A simple example is propositional logic --- the connectives
$\land$, $\lor$ and $\to$ are represented as types, and a theorem
prover is used to prove logical formulae.  Having an implementation of
type theory and an interactive theorem prover accessible as an API
makes it easy to write tools for working in a formal system, whether
for educational or practical purposes.  In section \ref{example1}, I
will give details of an implementation of propositional logic.

\end{description}

In general, the library can be used wherever formally certified code
is needed --- evaluation of dependently typed \Ivor{} programs is
possible from Haskell programs and the results can be inspected
easily.  Domain specific tactics are often required; e.g. an
implementation of a programming language with subtyping may require a
tactic for inserting coercions, or a computer arithmetic system may
require an implementation of Pugh's Omega decision
procedure~\cite{pugh-omega}.  \Ivor{}'s API is designed to make
implementation of such tactics as easy as possible.

In \Ivor{}'s dependent type system, types may be predicated on
arbitrary values. Programs and properties can be
expressed within the same self-contained system --- properties are
proved by construction, at the same time as the program is
written. The tactic language can thus be used not only for
constructing proofs but also for interactive program development.

%\subsection{Why Do We Need Another Theorem Prover?}

%Relationship to e.g. \Coq{}.

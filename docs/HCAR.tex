\documentclass{article}

\usepackage{hcar}
\usepackage{url}

\begin{document}

\begin{hcarentry}{Ivor}
\report{Edwin Brady}
\status{Active Development}
\makeheader

Ivor is a tactic-based theorem proving engine with a Haskell API. Unlike
other systems such as Coq and Agda, the tactic engine is primarily
intended to be used by programs, rather than a human operator. To this
end, the API provides a collection of primitive tactics and
combinators for building new tactics. This allows easy construction of
domain specific tactics, while keeping the core type theory small and
independently checkable.

The primary aim of the library is to support research into generative
programming and resource bounded computation in Hume
(\url{http://www.hume-lang.org/}). In this setting, we have developed a
dependently typed framework for representing program execution cost,
and used the Ivor library to implement domain specific tactics for
constructing programs within this framework. However the library is
more widely applicable, some potential uses being:

\begin{itemize}
\item A core language for a richly typed functional language.
\item The underlying implementation for a theorem prover (see first order
  logic theorem prover example at
  \url{http://www.dcs.st-and.ac.uk/~eb/Ivor}).
\item An implementation framework for a domain specific language requiring
  strong correctness properties.
\end{itemize}

Ivor features a dependent type theory similar to Luo's ECC with
definitions, with additional (experimental) multi-stage programming
support. Optionally, it can be extended with heterogenous equality,
primitive types and operations, new parser rules and user defined
tactics. By default, all programs in the type theory terminate, but in
the spirit of flexibilty, the library can be configured to allow
general recursion.

The library is in active development, although at an early
stage. Future plans include development of more basic tactics (for
basic properties such as injectivity and disjointness of constructors,
and elimination with a motive), a compiler (with optimisations) and a
larger collection of standard definitions.

\FurtherReading
  \url{http://www.dcs.st-and.ac.uk/~eb/Ivor}
\end{hcarentry}

\end{document}
